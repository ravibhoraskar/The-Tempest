\documentclass{beamer}
\usepackage[latin1]{inputenc}
\usepackage{amsfonts}
\usepackage{epsfig}
\usepackage{hyperref}
\usepackage{multicol}
\usepackage{graphicx}
\usetheme{Warsaw}
\title[The Tempest \hspace{15em} \insertframenumber / \inserttotalframenumber]{Colonization in The Tempest and its Afterlives}

\author {Ravi Bhoraskar}
\begin{document}
\begin{frame}[plain]
\titlepage
\begin{center}
\begin{figure}[htp]
  \begin{center}
    \centering
    \includegraphics[scale=0.17]{title.jpg}
  \end{center}
\end{figure}
under guidance of\\
Prof. Sudha Shastri\\
IIT Bombay
\end{center}
\end{frame}
\begin{frame}{Outline}
  \begin{multicols}{2}
  \tableofcontents  
  \end{multicols}
\end{frame}
\section{Introduction}
\subsection{A brief history of Colonization}
\begin{frame}{A brief history of British Colonization}
  \begin{columns}[c]
    \column{.6\textwidth}
  \begin{itemize}
  \item \textbf{1492} Columbus discovered America (mistook it for India)
  \item \textbf{1497} Henry VII commissions John Cabot; First Englishman in America
   %No colonization so far, just trade. John Cabot's ship disappeared in his second voyage.
  \item \textbf{1576 onwards} Early claims of land in the name of queen
  \item \textbf{1607} Colonization of America started, with \emph{Virginia Company} creating colony at Jamestown, Virginia
    %Parallels with the colonial discourse prophetic, rather than descriptive
  \end{itemize}
  \column{.4\textwidth}
    \begin{figure}[htp]
      \begin{center}
        \centering
        \includegraphics[scale=0.29]{columbus.jpg}
      \end{center}
    \end{figure}
    \end{columns}
\end{frame}

\begin{frame}{The Sea Venture incident}
  \begin{columns}[c]
    \column{.4\textwidth}
    \begin{figure}[htp]
      \begin{center}
        \centering
        \includegraphics[scale=0.29]{seaventure.jpg}
      \end{center}
    \end{figure}
    \footnotesize{\emph{The Sea Venture in a heavy Sea in 1609,} painting by Christopher Grimes}
    \column{.6\textwidth}

  \begin{itemize}
  \item \emph{Sea Venture} commissioned in 1609 to deliver supplies to the Jamestown colony
  \item Set sail on June 20; Ran into storm and sank on July 24
  \item All 150 passengers land safely ashore, onto a reef in Bermuda
  \item Stranded for 9 months; built 2 new boats; sailed to Virginia; most saved
  \item William Strachey wrote accounts, which supposedly inspired The Tempest
  \end{itemize}
  \end{columns}
\end{frame}


\begin{frame}{The Tempest}
  See video 'The Tempest in a minute'
\end{frame}

\subsection{The Tempest}
\begin{frame}{The Tempest}
  \begin{itemize}
    \item Shakespeare's last play; written in 1610-11
    \item Strict observation of the three unities, in contrast to other Shakespeare's plays 
    \item Prospero an image of Shakespeare, with his renunciation of magic depicting Shakespeare's farewell to the stage
    \item Colonization had just started, hence parallels with the colonial discourse prophetic, rather than descriptive
  \end{itemize}
\end{frame}

\subsection{The Afterlives}
\begin{frame}{The Afterlives}
  \begin{enumerate}
  \item \textbf{Neil Gaiman's \emph{The Tempest}}
    \begin{itemize}
    \item \textbf{1997} Last book of \emph{The Sandman} series, part of \emph{The Wake}
    \end{itemize}
  \item \textbf{Charles and Mary Lamb's \emph{Tales from Shakespeare}}
    \item \textbf{1807} Written as an abridged version of Shakespeare for children
  \item \textbf{The Tempest: Graphic Novel}
    \begin{itemize}
    \item \textbf{2009} Retains the original text, but adds pictures
    \end{itemize}
  
  \end{enumerate}
\end{frame}


\end{document}
